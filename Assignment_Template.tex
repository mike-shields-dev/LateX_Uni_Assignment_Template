\documentclass[12pt]{article}

% Set page margins
\usepackage[a4paper, margin=0.8in]{geometry}

% Use XeLaTeX or LuaLaTeX to handle system fonts
\usepackage{fontspec}
\setmainfont{Calibri Light} % Set main font to Calibri Light

% Set line spacing to double
\usepackage{setspace}
\doublespacing

% Format section headings with titlesec
\usepackage{titlesec}

% Set paragraph indentation to 0pt
\setlength{\parindent}{0pt}
\setlength{\parskip}{14pt}

% Define section format
\titleformat{\section}
  [hang]
  {\fontsize{18pt}{14pt}\bfseries\leftskip=0pt}
  {\thesection}
  {1em}
  {}

% Define subsection format
\titleformat{\subsection}
  [hang]
  {\fontsize{16pt}{14pt}\bfseries\leftskip=0pt}
  {\thesubsection}
  {1em}
  {}

% Define subsubsection format
\titleformat{\subsubsection}
  [hang]
  {\fontsize{14pt}{14pt}\bfseries\leftskip=0pt}
  {\thesubsubsection}
  {1em}
  {}

% Include xcolor package for hyperref color options
\usepackage{xcolor}

% Set up the fancyhdr package
\usepackage{fancyhdr}
\pagestyle{fancy}
\fancyhf{} % Clear existing header and footer
\fancyfoot[C]{Page \thepage\ of \pageref{LastPage}} % Center of footer with page number

% Remove the horizontal line below the header
\renewcommand{\headrulewidth}{0pt}

% For page number references
\usepackage{lastpage} % For page number references

% For cross-references and clickable links
\usepackage[
  colorlinks=true, % Enable colored links
  linkcolor=black,  % Color for internal links
  urlcolor=black,   % Color for URLs
  citecolor=black,  % Color for citations
  filecolor=black,  % Color for file links
  linkbordercolor=black, % Color for link borders
  urlbordercolor=black,  % Color for URL borders
]{hyperref} % Must be included after most other packages

\usepackage{bookmark} % For better bookmark handling

% Increase table padding
\setlength{\tabcolsep}{12pt} % Horizontal padding
\renewcommand{\arraystretch}{1.5} % Vertical padding

\begin{document}

\begin{table}[h]
  \centering
  \begin{tabular}{|l|l|}
      \hline
      Course / Programme: & BEng Software Engineering \\
      \hline
      Module name and code: & .......... \\
      \hline
      Student ID: & 2324915 \\
      \hline
      Tutor: & ............. \\
      \hline
      Assignment Number: & 0 \\
      \hline
      Assignment Title: & ........... \\
      \hline
      Weighting: & 0\% of overall grade \\
      \hline
      Issue Date: & 00/00/0000 \\
      \hline
      Submission Deadline: & 00/00/0000 @00:00 \\
      \hline
      \multicolumn{2}{|l|}{
        \begin{minipage}{\linewidth}
        \vspace{16pt} % Add vertical space to ensure padding
        Learning Outcomes Assessed:
          \begin{enumerate}
            \item Outcome 1
            \item Outcome 2
            \item Outcome 3
          \end{enumerate}
          \vspace{16pt} % Add vertical space to ensure padding
        \end{minipage}
      } \\
      \hline
  \end{tabular}
\end{table}

\newpage

% Table of Contents
\tableofcontents

\newpage % Start new page after Table of Contents

\section{Introduction}
\label{sec:introduction}
This is an introduction section with left-aligned headings. This text is aligned left for headings and justified for paragraphs. This is an example of a paragraph to demonstrate the justification. Hello.

\subsection{Background}
\label{sec:background}
See Section~\ref{sec:introduction} for the introduction.
See Section~\ref{sec:details} for the details.
See Section~\ref{sec:another-section} for another section.

Here is an example of a bulleted list:
\begin{itemize}
  \item Item 1
  \item Item 2
  \item Item 3
\end{itemize}

\subsubsection{Details}
\label{sec:details}
See Section~\ref{sec:introduction} for the introduction.
See Section~\ref{sec:background} for the background.
See Section~\ref{sec:another-section} for another section.

\section{Another Section}
\label{sec:another-section}
See Section~\ref{sec:introduction} for the introduction.
See Section~\ref{sec:background} for the background.
See Section~\ref{sec:details} for the details.

\label{LastPage}
\end{document}
