\documentclass[12pt]{article}

% Document Macros
\newcommand{\studentid}{123456}

% Include the preamble
% Bibliography setup
\usepackage[backend=biber, style=authoryear, sorting=ynt]{biblatex}
\addbibresource{references.bib}

% Set page margins
\usepackage[a4paper, margin=1in]{geometry}

% Use XeLaTeX or LuaLaTeX to handle system fonts
\usepackage{fontspec}
\setmainfont{Calibri Light} % Set main font to Calibri Light

% Set line spacing to double
\usepackage{setspace}
\doublespacing

% Justify text
\usepackage{ragged2e}
\justifying

% Format section headings with titlesec
\usepackage{titlesec}

% Set paragraph indentation to 0pt
\setlength{\parindent}{0pt}
\setlength{\parskip}{14pt}

% Define section format
\titleformat{\section}
  [hang]
  {\fontsize{18pt}{14pt}\bfseries\leftskip=0pt}
  {\thesection}
  {1em}
  {}

% Define subsection format
\titleformat{\subsection}
  [hang]
  {\fontsize{16pt}{14pt}\bfseries\leftskip=0pt}
  {\thesubsection}
  {1em}
  {}

% Define subsubsection format
\titleformat{\subsubsection}
  [hang]
  {\fontsize{14pt}{14pt}\bfseries\leftskip=0pt}
  {\thesubsubsection}
  {1em}
  {}

% Set up the fancyhdr package
\usepackage{fancyhdr}
\pagestyle{fancy}
\setlength{\headheight}{15pt} % Adjust the header height
\fancyhf{} % Clear existing header and footer
\fancyhead[L]{Student ID: 2324915} % Left of header with section name
\fancyfoot[C]{Page \thepage\ of \pageref{LastPage}} % Center of footer with page number

% Remove the horizontal line below the header
\renewcommand{\headrulewidth}{0pt}

% Set up the tocloft package
\usepackage{tocloft}
\setlength{\cftbeforesecskip}{0pt} % Adjust space between sections in the table of contents

% For page number references
\usepackage{lastpage} % For page number references

% Include xcolor package for hyperref color options
\usepackage{xcolor}

% For cross-references and clickable links
\usepackage[
  colorlinks=true, % Enable colored links
  linkcolor=black,  % Color for internal links
  urlcolor=black,   % Color for URLs
  citecolor=black,  % Color for citations
  filecolor=black,  % Color for file links
  linkbordercolor=black, % Color for link borders
  urlbordercolor=black,  % Color for URL borders
]{hyperref} % Must be included after most other packages

\usepackage{bookmark} % For better bookmark handling


% Increase table padding
\setlength{\tabcolsep}{12pt} % Horizontal padding
\renewcommand{\arraystretch}{1.5} % Vertical padding


%TC:ignore

% Document Metadata
\title{UoB LaTex Assignment Template & Guide} 

%TC:endignore

\begin{document}


%TC:ignore
\newpage

%TC:ignore
\begin{table}[h]
    \centering
    \begin{tabular}{|l|l|}
        \hline
        Course / Programme: & BEng Software Engineering \\
        \hline
        Module Name and Code: & .......... \\
        \hline
        Student ID: & \studentid \\
        \hline
        Tutor: & ............. \\
        \hline
        Assignment Number: & 0 \\
        \hline
        Assignment Title: & ........... \\
        \hline
        Weighting: & 0\% of overall grade \\
        \hline
        Issue Date: & 00/00/0000 \\
        \hline
        Submission Deadline: & 00/00/0000 @00:00 \\
        \hline
        \multicolumn{2}{|l|}{
          \begin{minipage}{\dimexpr\textwidth-2\tabcolsep\relax} % Adjust width to fit the table
          \vspace{8pt} % Add vertical space to ensure padding
          Learning Outcomes Assessed:
            \begin{enumerate}
              \item Outcome 1
              \item Outcome 2
              \item Outcome 3
            \end{enumerate}
            \vspace{8pt} % Add vertical space to ensure padding
          \end{minipage}
        } \\
        \hline
    \end{tabular}
  \end{table}
  %TC:endignore

  \newpage

\tableofcontents
%TC:endignore


%TC:ignore
\section{Introduction}
\label{sec:introduction}
%TC:endignore

This document serves as a template for producing written assignments in accordance with the University of Bolton's assignment format specifications.

The subject material of this document serves as a guide demonstrating how to use various \LaTeX{} features to address common requirements for academic writing, such as:

\begin{itemize}
  \item Table of Contents
  \item Document Structure
  \item Lists
  \item Cross-references / Internal Hyperlinks
  \item External Hyperlinks
  \item References and Citations
  \item Mathematical Expressions
  \item Computer Source Code
\end{itemize}

%TC:ignore
\section{What is LaTeX?}
\label{sec:what-is-latex}
%TC:endignore

\LaTeX{} is a typesetting system commonly used for producing scientific and technical documents. It is widely used in academia for writing research papers, theses, and other technical documents. 

To learn more about \LaTeX{}, visit the \href{https://www.latex-project.org/}{\LaTeX{} Project Website}. 

Other useful resources include \cite{mittelbach_2023_the}

%TC:ignore
\subsubsection{The Anatomy of a LaTeX Document}
\label{sec:the-anatomy-of-a-latex-document}
%TC:endignore

A \LaTeX{} document consists of two main parts: the \textbf{preamble} and the \textbf{document body}. The preamble contains document-wide settings and configurations, while the document body contains the actual content of the document.

The preamble includes aspects of the document such as:

\begin{itemize}
  \item Document class declaration
  \item Packages and libraries and external resources
  \item Custom commands and settings
  \item Title, author, and date information
\end{itemize}

%TC:ignore
\section{UoB Assignment Format Specifications}
\label{sec:uob-assignment-format-specifications}
%TC:endignore

The specifications for the University of Bolton's assignment format are as follows:

\begin{itemize}
  \item Base font: \textbf{Calibri Light} (12pt)
  \item Double line spacing
  \item \textbf{Harvard} style referencing
  \item Page numbers in the format: \textbf{"Page n of m"}
  \item A table of contents
  \item A title page
  \item In text citations
  \item A word count at the end of the document
  \item A bibliography
\end{itemize}

%TC:ignore
\subsubsection{Font}
\label{sec:details}
%TC:endignore

The base font for the document: \textbf{Calibri Light} is not a font that is typically available in \LaTeX{} distributions. However, it is possible to use \textbf{installed} system fonts with \LaTeX{} by using the \texttt{fontspec} package.

\textit{ Note: The \textbf{Calibri Font Family} ttf files are located in the \texttt{\href{https://github.com/mike-shields-dev/University_of_Bolton_LaTeX_Assignment_Template/tree/main/assets}{assets}} directory of this document's parent github repository.} 


The following code snippet demonstrates how to set the main font to \textbf{Calibri Light}:
\begin{verbatim}
\usepackage{fontspec}
\setmainfont{Calibri Light}
\end{verbatim}

%TC:ignore
\section{Another Section}
\label{sec:another-section}
%TC:endignore

\newpage

%TC:ignore
\section{Conclusion}
\label{sec:conclusion}
%TC:endignore

\newpage

%TC:ignore
\section{Word Count}
\label{sec:word-count}
%TC:endignore

The word count for this document is: \textbf{[word count]}.



\newpage

% Print the bibliography
\printbibliography

\end{document}
